%%	SECCION documentclass													 %%
%%---------------------------------------------------------------------------%%
\documentclass[a4paper]{article}
%%---------------------------------------------------------------------------%%
%%	SECCION usepackage														 %%	
%%---------------------------------------------------------------------------%%
\usepackage{amsmath, amsthm}
\usepackage{amsfonts}
\usepackage{amssymb}
\usepackage{0_caratula/caratula}
\usepackage{hyperref}
\usepackage{graphicx}
\usepackage{alltt}
\usepackage[spanish]{babel} 
\usepackage[T1]{fontenc}
\usepackage{textcomp}
\usepackage[utf8]{inputenc}
\usepackage{fancyhdr}
\usepackage{vmargin}

\renewcommand{\thesection}{\arabic{section}}
\renewcommand{\headrulewidth}{0.4pt}
\renewcommand{\footrulewidth}{0.4pt}
\setmarginsrb
{15mm}% left margin
{10mm}% top margin
{15mm}% right margin
{10mm}% bottom margin
{0mm}{10mm}{0mm}{10mm}% we needed -- related to headers and footers

%\makeindex

%%---------------------------------------------------------------------------%%
%%	SECCION document	                                                     %%
%%---------------------------------------------------------------------------%%
\begin{document}

\setcounter{page}{0}
\pagestyle{fancy}
\fancyhead{}
\fancyfoot{}
\fancyhead[L]{\footnotesize Borghini, Lamelas Marcote, Pérez}
\fancyhead[R]{\footnotesize Investigación Operativa}
\fancyfoot[C]{\footnotesize \thepage}


%%---- Caratula -------------------------------------------------------------%%
\title{Investigación Operativa}
\author{Matías Pérez\thanks{This is for making an acknowledgement.}
\\Universidad de Buenos Aires, Argentina}
\date{14 April, 2009}

\materia{Investigación Operativa}
\titulo{Trabajo Práctico}
\subtitulo{Resolución de problemas lineales binarios.}
\fecha{Diciembre 2012}


\integrante{Fabrizio Borghini}{???/??}{fabriborghini@gmail.com}
\integrante{Sebastián Rodrigo Lamelas Marcote}{153/08}{cobycobz@msn.com}
\integrante{Matías Pérez}{002/05}{elmaildematiaz@gmail.com}
%\grupo{}
%\nombreGrupo{}
%\resumen{}

\maketitle

\tableofcontents

\newpage


\section{Introducción}

\subsection{Esquema de los algoritmos}

En este trabajo práctico se utiliza el software \emph{IBM\textsuperscript{\textregistered} ILOG\textsuperscript{\textregistered} CPLEX\textsuperscript{\textregistered}} como motor principal para desarrollar un $resolver$ de problemas lineales con variables enteras binarias (todas las variables del problema toman el valor $\in \{0,1\}$). Los problemas de este tipo suelen llamarse $MIP$s (\emph{Mixed Integer Problems}). En particular, en este trabajo, se utilizan tres tipo de técnicas diferentes para resolverlos:

\begin{itemize}
\item \emph{Branch \& Bound} ($BB$):\\
Se resuelve la relajación lineal del MIP (es decir, se ignoran las condiciones de integralidad de las variables, asumiendo que vale tener variables fraccionarias). Se toma una variable $x_i$ fraccionaria con valor $x^*_i$ en la relajación. A partir de ella se desprenden dos subproblemas (\emph{branching}) que dejan fuera la posibilidad de asignar nuevamente $x^*_i$ a $x_i$. El primer subproblema es igual al resuelto agregando al restricción $x_i \leq \lfloor x^*_i \rfloor$. El segundo subproblema es igual al resuelto agregando la restricción $x_i \geq \lceil x^*_i \rceil$. A partir de estos dos problemas se repite el procedimiento hasta encontrar la mejor solución que cumpla las condiciones de integralidad o deducir que ésta no existe.

\item \emph{Cut \& Branch} ($CB$):\\
Es análogo al $BB$ pero al analizar por primera vez el MIP en vez de resolver y hacer $branching$, se obtiene la solución óptima de la relajación $x^*$ y se buscan planos de corte (nuevas restricciones que sean válidas para soluciones enteras del MIP pero dejen afuera la solución $x^*$). Si se encuentran, se vuelve a resolver el MIP original (esto hace que la nueva solución óptima sea diferente de $x^*$) y si la nueva solución óptima no es entera se buscan nuevamente planos de corte. Si no se encuentran (o si ya se encontraron varios planos de corte y es preferible hacer $branching$), comienza a subdividirse al MIP como en $BB$ hasta encontrar la solución óptima entera o deducir que no puede hallarse la misma.

\item \emph{Branch \& Cut} ($BC$):\\
Es análogo al $CB$ porque la primera vez que se resuelve el MIP también se usa la estrategia de los planos de corte. Sin embargo, a diferencia del $CB$, esta estrategia la utiliza también en los subproblemas originados gracias al $branching$. O sea, no importa si está resolviendo por primera vez el MIP o está resolviendo un subproblema derivado del mismo a través de una serie de $branching$s: siempre va a buscar planos de corte, y volver a resolver la relajación con los planos de corte agregados. Después de haber encontrado una cantidad de cortes igualmente se sigue haciendo $branching$ hasta encontrar la solución entera óptima o deducir que no existe.
\end{itemize}

Es necesario realizar una observación. Las tres técnicas continúan generando subproblemas hasta ``encontrar la mejor solución o deducir que ésta no existe". Lo que quiere decir esto es que a medida que se generan subproblemas puede o no encontrarse una solución entera válida. Si no se encuentra ninguna es porque no existe. Y, cuando existe una, para saber que es la óptima hay que terminar de resolver todos los subproblemas derivados del $branching$. Es decir, el espacio de búsqueda de la solución es ``muy grande'' porque hay que resolver todos los subproblemas derivados a partir del MIP original obligatoriamente (para comprobar la no existencia o la optimalidad).\\

La idea en las tres téncicas es la misma siempre. Resolver un problema con un algoritmo aproximado (la relajación lineal) y continuamente tratar de mejorar la calidad de la solución encontrada hasta que cumpla condiciones de integralidad. El proceso de $branching$ deja afuera la solución fraccionaria encontrada porque parte la variable en dos zonas. El proceso de planos de corte también deja afuera la solución fraccionaria. Entonces, puede verse que con ambas estrategias el espacio de búsqueda de la solución va achicándose. Esta demostración intuitiva ayuda a comprender por qué eventualmente se llega a una solución entera válida.

\newpage

\subsection{Estrategias de Planos de Corte}

Los algoritmos $CB$ y $BC$ usan ``planos de corte'' para encontrar restricciones que sean válidas para cualquier solución entera del MIP pero que no sean válidas para algunas soluciones fraccionarias (al menos, que no sea válida para la solución de la relajación). Esta estrategia tiene una velocidad de convergencia más lenta que el $BB$, pero la experiencia demuestra que la combinación de los planos de corte con el $branching$ suele dar mejor eficiencia para resolver un MIP que sólo con el $BB$. Por esta razón se agregan planos de corte en cada subproblema derivado del MIP y por esta razón igualmente se sigue haciendo branching. La línea que dibuja el límite en la relación entre la cantidad de planos de corte y la cantidad de $branching$ es bastante difusa y cada optimización tiene resultados distintos.\\

Con respecto a los planos de corte, existen diversas formas de obtenerlos. En particular, en este trabajo, se utilizarán dos tipos:

\begin{itemize}
\item \emph{Cortes Cover}
\item \emph{Cortes Clique}
\end{itemize}

En la próxima sección se eplican estos dos tipos de corte y brevemente se describen algunos detalles de implementación. Con ambos cortes explicados y las tres técnicas algorítmicas distintas, se mostrarán resultados obtenidos de correr una serie de casos de prueba con la implementación realizada. Finalmente, se intentarán sacar conclusiones o presentar observaciones acerca de los resultados, implementaciones y conceptos explicados.

\newpage
\section{Explicación}

\subsection{Cortes Cover}
\subsubsection{Cortes Cover: Explicación Teórica}

\subsubsection*{Definicion}
Una ``desigualdad mochila'' es una desigualdad con $a_j \geq 0$ ($1\leq j \leq n$) de la forma:

$$ \overset{n}{\underset{j=1}{\sum}} a_j\,x_j \leq b$$

En este ipo de desigualdades se pude buscar un ``cover''. Es decir, un conjunto $C \subseteq \{1,...,n\}$ de índices tales que:

$$ \overset{}{\underset{j \in C}{\sum}} a_j\,x_j > b$$

Si se encuentra el mismo, entonces se obtiene una nueva ``desigualdad cover'' válida:

$$ \overset{}{\underset{j \in C}{\sum}} x_j \leq |C| - 1$$

Donde $|C|$ es el cardinal del cover $C$ encontrado.

\subsubsection*{Corte Cover}

Un ``corte cover'', entonces, consiste en encontrar una desigualdad cover válida para toda solución entera del MIP pero que deje afuera soluciones fraccionarias. La idea es recibir una solución óptima de la relajación de un nodo $x^*$ como input y a partir de ella fabricar una desigualdad cover violada por $x^*$:\\

{
\centering
\begin{tabular}{c l}
\verb_INPUT_ & $x^*$\\
\verb_OUTPUT_ & Desigualdad cover violada por $x^*$\\
\end{tabular}\\
\vspace{5mm}
}

Para encontrar el corte cover en este trabajo se siguen los siguientes pasos:

\begin{enumerate}
\item Tomar una restricción original del MIP.
\item Traducirla a desigualdad mochila.
\item Búsqueda del cover.
\item Extensión del cover.
\item Reescribir el cover en función de variables originales.
\end{enumerate}

A continuación, la explicación detallada de cada paso.
\newpage

\begin{enumerate}
\item \underline{Tomar una restricción original del MIP:}\\
Cualquier restricción de la forma:
$$\overset{n}{\underset{j=1}{\sum}} a_j\;x_j \leq b$$

O, alternativamente, tomar una restricción de la siguiente forma:
$$\overset{n}{\underset{j=1}{\sum}} a_j\;x_j \geq b$$

Y multpilicar ambos lados por $-1$ para tener la forma de la primera inecuación (con ``$\leq$'').

\item \underline{Traducirla a desigualdad mochila:}\\
Con el conocimiento que las variables son binarias ($x_j\in\{0,1\}$) el primer paso consiste en traducir cada restricción original (con ``$\leq$'') del MIP en una desigualdad mochila. Sean los conjuntos:

$$N_+ = \{j: a_j > 0 \;\;\forall j\;\; 1\leq j \leq n\}$$
$$N_- = \{j: a_j < 0 \;\;\forall j\;\; 1\leq j \leq n\}$$

Y usando el cambio de variables:

$$x_j + \bar{x}_j = 1$$

Se puede hacer una traducción de la siguiente forma:

$$\overset{n}{\underset{j=1}{\sum}} a_j\;x_j \leq b$$
$$\Leftrightarrow$$
$$\overset{}{\underset{j \in N_+}{\sum}} a_j\;x_j - \overset{}{\underset{j \in N_-}{\sum}} |a_j|\;x_j \leq b$$
$$\Leftrightarrow$$
$$\overset{}{\underset{j \in N_+}{\sum}} a_j\;x_j - \overset{}{\underset{j \in N_-}{\sum}} |a_j|\;(1-\bar{x}_j) \leq b$$
$$\Leftrightarrow$$
$$\overset{}{\underset{j \in N_+}{\sum}} a_j\;x_j + \overset{}{\underset{j \in N_-}{\sum}} |a_j|\;\bar{x}_j \leq b + \overset{}{\underset{j \in N_-}{\sum}} |a_j|$$

De esta forma, en el caso de restricciones con variables binarias, cualquier inecuación puede traducirse a desigualdad mochila:

{
\centering
\begin{tabular}{p{4cm}p{4cm}p{4cm}}
$\overset{}{\underset{j \in N_+ \cup N_-}{\sum}} |a_j|\;\tilde{x}_j \leq \tilde{b}$
&
$\tilde{b} = b + \overset{}{\underset{j \in N_-}{\sum}} |a_j|$
&
$\tilde{x}_j = 
\begin{cases}
x_j & \text{ si } j \in N_+\\
\bar{x}_j & \text{ si } j \in N_-\\
\end{cases}$\\
\end{tabular}\\
\vspace{5mm}
}

\item \underline{Búsqueda del cover:}\\
Dada $x^*$ con $0\leq x_j \leq 1$ ($1\leq j \leq n$) solución óptima de la relajación, se busca un cover $C\subseteq \{1,...,n\}$ tal que:

{
\centering
\begin{tabular}{ccc}
$\overset{}{\underset{j \in C}{\sum}} x^*_j > |C|-1$ & y & $\overset{}{\underset{j \in C}{\sum}} \lfloor \tilde{a}_j \rfloor > \lceil \tilde{b} \rceil$\\
\end{tabular}\\
\vspace{5mm}
}

Este cover $C$ puede encontrarse resolviendo el MIP:

{
\centering
\begin{tabular}{rl}
minimizar & $\overset{n}{\underset{j=1}{\sum}} (1 - x^*_j) y_j$\\
sujeto a&\\
&$\overset{n}{\underset{j=1}{\sum}} \lfloor \tilde{a}_j \rfloor y_j \geq \lceil \tilde{b} \rceil + 1$\\
&$y_j \in \{0,1\}$\\
\end{tabular}\\
\vspace{5mm}
}

Si el funcional es mayor o igual a 1, no existe un cover violado por $x^*$. Si se encuentra $y^*$ tal que el valor del óptimo es menor a 1, los valores de $y^*$ hacen un cover violado. Sea $G=\{j: y^*_j = 1\}$ ($G$ es el cover $C$ buscado):

{
\centering
\begin{tabular}{cc}
$\overset{}{\underset{j \in G}{\sum}} \tilde{x}_j \leq |G|-1$ & $(< \overset{}{\underset{j \in G}{\sum}} x^*_j)$\\
\end{tabular}\\
\vspace{5mm}
}

Esta desigualdad es válida porque como muestra el MIP que la encuentra, es una violación a la restricción original del problema; y, además, la desigualdad válida es un plano de corte porque al agregarla se descarta la solución óptima $x^*$ ya que ésta viola el cover encontrado.

\item \underline{Extensión del cover:}

Una vez obtenido el cover $G$:

$$\overset{}{\underset{j \in G}{\sum}} \tilde{x}_j \leq |G|-1$$

Se puede extender de la siguiente forma:

$$E(G) = G \cup \{j: \tilde{a}_j \geq \tilde{a}_k \;\;\forall k\;\; \in G\}$$

Y queda el cover extendido:

$$\overset{}{\underset{j \in E(G)}{\sum}} \tilde{x}_j \leq |G|-1$$

\item \underline{Reescribir el cover en función de variables originales:}\\
Una vez que se consiguió un cover hay que recuperar las variables originales usando el reemplazo:

$$\tilde{x}_j = 
\begin{cases}
x_j & \text{ si } j \in N_+\\
1-x_j & \text{ si } j \in N_-\\
\end{cases}$$

Entonces:

$$\overset{}{\underset{j \in E(G)}{\sum}} \tilde{x}_j \leq |G| - 1$$
$$\Leftrightarrow$$
$$\overset{}{\underset{j \in E(G)\cap N_+}{\sum}}x_j + \overset{}{\underset{j \in E(G)\cap N_-}{\sum}} (1-x_j) \leq |G|-1$$

El cover quedará finalmente:

$$\overset{}{\underset{j\in E(G)}{\sum}} \tilde{x}_j \leq |G|-1$$
$$\Leftrightarrow$$
$$\overset{}{\underset{j\in E(G)\cap N_+}{\sum}} x_j - \overset{}{\underset{j \in E(G)\cap N_-}{\sum}} x_j \leq |G| - h -1$$

Con $h = |E(G)\cap N_-|$.
\end{enumerate}

\newpage
\subsubsection{Cortes Cover: Detalles de Implementación}

\newpage

\subsection{Cortes Clique}
\subsubsection{Cortes Clique: Explicación Teórica}

La idea central de este método es armar un grafo, llamado \textit{grafo de conflictos}, que respresenta relaciones l\'ogicas entre las variables del problema lineal. Dicho grafo tiene un v\'ertice por cada variable ($x_i$) y su complemento ($\bar{x}_i = 1 - x_i$), y un eje entre dos v\'ertices cuando a lo sumo una de las dos variables que comparten el eje puede estar en 1 en una soluci\'on \'optima. Hay cuantro relaciones l\'ogicas entre las variables (recordemos que las variables son binarias):
\begin{eqnarray*}
x_i = 1 \Rightarrow x_j = 0 & \Longleftrightarrow & x_i + x_j \leq 1 \\
x_i = 0 \Rightarrow x_j = 0 & \Longleftrightarrow & (1 - x_i) + x_j \leq 1 \\
x_i = 1 \Rightarrow x_j = 1 & \Longleftrightarrow & x_i + (1 - x_j) \leq 1 \\
x_i = 0 \Rightarrow x_j = 1 & \Longleftrightarrow & (1 - x_i) + (1 - x_j) \leq 1 
\end{eqnarray*}

Seguro una variable y su complemento est\'an unidos por un eje ya que exactamente una de las dos tiene que ser igual a 1 (en realidad es una relaci\'on m\'as fuerte). Entonces, se construye un grafo $G = (V,E)$ utilizando t\'ecnicas para deducir relaciones entre variables. Estas t\'ecnicas son llamadas de \textit{probing}: b\'asicamente es setear alguna variable binaria en una de sus cotas y estudiar las consecuencias. Si se asigna una variable en una de sus cotas y el problema se vuelve infactible, esto quiere decir que la variable necesariamente tiene que estar en la cota opuesta.

Se van a desarrollar brevemente dos t\'ecnicas utilizadas en el trabajo para deducir relaciones entre variables para agregar ejes al \textit{grafo de conflictos}. Una primera fase relativamente simple para agregar ejes y una segunda fase m\'as fuerte para intentar completar con m\'as ejes.

\subsubsection*{Primera fase}

Sea $S = \{x\in \{0,1\}^n : Ax \leq b\}$ donde $Ax \leq b$ representa las restricciones lineales del problema a resolver. Se define el siguiente conjunto:
$$S_{x_i=v_i,x_j=v_j} = \{x\in S : x_i = v_i, x_j = v_j\}$$ 
donde $v_i,v_j\in \{0,1\}$ y $x_i \neq x_j$. Es decir, las soluciones del problema tales que $x_i$ y $x_j$ est\'an seteados en $v_i$ y $v_j$ respectivamente. Como calcular este conjunto es tan dif\'icil como resolver el problema original, se trabaja con la relajaci\'on $S'_{x_i=v_i,x_j=v_j}$ de $S_{x_i=v_i,x_j=v_j}$ (pues $S'_{x_i=v_i,x_j=v_j} \supseteq S_{x_i=v_i,x_j=v_j}$). \\

Luego, sea $L^r_{x_i=v_i,x_j=v_j} = min\{a_rx : x\in S'_{x_i=v_i,x_j=v_j}\}$ donde $a_r$ es la \textit{r-\'esima} fila de $A$. De $L^r$ se puede deducir lo siguiente:
\begin{itemize}
\item $L^r_{x_i=1,x_j=1} > b_r \Longrightarrow x_i + x_j \leq 1$ es v\'alido para $S$ y $(i,j)\in E$
\item $L^r_{x_i=1,x_j=0} > b_r \Longrightarrow x_i + (1-x_j) \leq 1$ es v\'alido para $S$ y $(i,\bar{j})\in E$ 
\item $L^r_{x_i=0,x_j=1} > b_r \Longrightarrow (1-x_i) + x_j \leq 1$ es v\'alido para $S$ y $(\bar{i},j)\in E$ 
\item $L^r_{x_i=0,x_j=0} > b_r \Longrightarrow (1-x_i) + (1-x_j) \leq 1$ es v\'alido para $S$ y $(\bar{i},\bar{j})\in E$
\end{itemize}

Utilizar la relajaci\'on $S'_{x_i=v_i,x_j=v_j}$ puede ser costosa, as\'i que se puede usar otra relajaci\'on m\'as d\'ebil pero con un esfuerzo computacionalmente menor. La relajaci\'on utilizada es la siguiente: usar simplemente las cotas de las variables. Para cada fila $r$ se define: \\ 
$$B^+_r = \{j: a_{rj} > 0 \text{ con } 1\leq j \leq n\}$$
$$B^-_r = \{j: a_{rj} < 0 \text{ con } 1\leq j \leq n\}$$

Entonces, se toma la siguiente relajaci\'on: 

$$L^r_{x_i=v_i,x_j=v_j} = \overset{}{\underset{k\in B^-_r,k\neq i,k\neq j}{\sum}} a_{rk} + a_{ri}v_i + a_{rj}v_j$$

As\'i, para cada restricci\'on $r$ se hacen todas las combinaciones con $x_i = v_i$ y $x_j = v_j$, $v_i,v_j\in \{0,1\}$ y se calcula el $L^r$ correspondiente para saber si $L^r > b_r$. En tal caso, se agrega el eje correspondiente al grafo de conflictos.

\subsubsection*{Segunda fase}

Utilizando las cotas de las variables y el grafo ya armado en la primera fase se puede conseguir una relajaci\'on m\'as fuerte. Se considera un subconjunto de v\'ertices $U \subseteq V$ que denotan las variables en $B^-_r$ y lo complementos de variables en $B^+_r$ de la fila $r$ de $A$. Definimos $w_k = |a_{rk}|$ para cada $k\in U$. Entonces, una cota inferior de $L^r$ para el m\'inimo valor del lado izquierdo de la fila $r$ se obtiene resolviendo el problema del conjunto independiente m\'as pesado del subgrafo inducido por los vértices de $U$:
\begin{center}
$\zeta^r = \sum_{k\in B^+_r}a_{rk} - max\{\sum_{k\in U}w_k\;z_k : z\quad$conjunto independiente de$\quad G(U)\}$
\end{center}

donde $G(U)$ es el subgrafo de $G$ inducido por $U$ y $z\in \{0,1\}^{|U|}$ el vector que caracteriza el conjunto independiente de $G(U)$ (es decir, $z_k = 1$ si $k$ pertenece al conjunto independiente y $z_k = 0$ si no). Se agrega la sumatoria de los $a_{rk}$ con $k\in B^+_r$ pues se usan los complementos de las variables con \'indices en $B^+_r$. 

Como resolver este nuevo problema es $NP$-hard, se resuelve una relajaci\'on del conjunto independiente m\'as pesado observando que si un grafo consiste en subgrafos completos disjuntos el conjunto independiente \'optimo se consigue eligiendo el v\'ertice m\'as pesado de cada subgrafo completo. Entonces, la idea es particionar los v\'ertices de $G(U)$ en un conjunto de cliques, utilizando alg\'un algoritmo que no sea muy costoso, y elegir el v\'ertice m\'as pesado de cada clique.

Una vez que la partici\'on en cliques está hecha, se busca un cota inferior para $L^r_{x_i=v_i,x_j=v_j}$ eligiendo el v\'ertice m\'as pesado de cada clique sujeto a que $x_i = v_i$ y $x_j = v_j$. Esto se hace para cada combinaci\'on de $v_i,v_j\in \{0,1\}$ teniendo cuidado de ignorar los casos en que haya un eje en el grafo de conflictos con dicha combinaci\'on.\\

{\footnotesize\underline{Observación:} para más detalles ver el \emph{paper} citado en $[1]$.}

\subsubsection*{Corte Clique}

Un ``corte clique'', consiste en encontrar una o m\'as cliques maximales en el grafo de conflictos armado que representen una desigualdad válida para toda solución entera del MIP pero que deje afuera soluciones fraccionarias. La idea es recibir la solución óptima de la relajación de un nodo $x^*$ como input y a partir de ella deducir una desigualdad clique violada por $x^*$:\\

{
\centering
\begin{tabular}{c l}
\verb_INPUT_ & $x^*$\\
\verb_OUTPUT_ & Desigualdad clique violada por $x^*$\\
\end{tabular}\\
\vspace{5mm}
}

Entonces la idea es buscar una (ó más) clique maximal $K$ en el grafo de conflictos tal que: 

$$\sum_{j\in K}\tilde{x}^*_j > 1$$

donde 
$\tilde{x}^*_j = 
\begin{cases}
\bar{x}^*_j & \text{ si j es complemento en el grafo}\\
x^*_j & \text{ si no}\\
\end{cases}$\\

Si se encuentra $K$, se puede agregar el siguiente corte clique v\'alido al conjunto de restricciones del MIP:

$$\sum_{j\in K}\tilde{x}_j \leq 1$$

Para encontrar un corte clique en este trabajo se siguen los siguientes pasos:

\begin{enumerate}
\item Armar el grafo de conflictos con los métodos antes explicados.
\item Buscar una o m\'as cliques maximales en el grafo, violadas por la relajaci\'on lineal.
\item Reescribir la desigualdad clique en función de las variables originales.
\end{enumerate}

A continuaci\'on se explica brevemente cada paso.
\newpage

\begin{enumerate}[ 1{)} ]
\item \underline{Armar el grafo de conflictos con las fases antes explicadas:}\\

Por cada restricci\'on del MIP se busca agregar ejes al grafo de conflictos probando cada combinaci\'on de $v_i$ y $v_j$ asignadas a cada variable con coeficiente distinto de 0 de la restricci\'on tanto en la primera fase como en la segunda. Luego de las dos fases queda armado el grafo de conflictos con las relaciones l\'ogicas v\'alidas entre las variables.

\item \underline{Buscar una o m\'as cliques maximales en el grafo, violadas por la relajaci\'on lineal:}\\

Con el óptimo de la relajaci\'on lineal obtenido y el grafo de conflictos armado, se buscan una o m\'as cliques maximales (con algún algoritmo heurísitico) tal que los valores de la relajaci\'on violen la desigualdad clique ($\sum_{j\in K}\tilde{x^*_j} > 1$ donde $K$ es la clique).

\item \underline{Reescribir la desigualdad clique en función de las variables originales:}\\

Una vez conseguido el corte clique (llamemos $K$ a la clique) hay que recuperar las variables originales usando el reemplazo:
$\tilde{x}_j = 
\begin{cases}
1-x_j & \text{ si } \bar{j}\in K\\
x_j & \text{ si } j\in K\\
\end{cases}$\\

Entonces:

$$\sum_{j\in K}\tilde{x}_j \leq 1$$
$$\Longleftrightarrow$$
$$\sum_{j\in K}x_j + \sum_{\bar{j}\in K}(1-x_j) \leq 1$$

El corte clique finalmente queda:

$$\sum_{j\in K}x_j - \sum_{\bar{j}\in K}x_j \leq 1 - k$$

donde $k = |\{\bar{j} : \bar{j}\in K\}|$

\end{enumerate}

\newpage
\subsubsection{Cortes Clique: Detalles de Implementación}

\newpage

\subsection{Más Comentarios}
\subsubsection{Más Comentarios: Teóricos}
Además de lo comentado anteriormente, hay que aclarar que a la hora de recorrer el árbol de subproblemas derivados del MIP original hay dos puntos clave en el que hay que elegir una estrategia de recorrido. El $CPLEX$ puede tener varios nodos (subproblemas derivados del MIP) activos (aún no se calculó su relajación lineal y no se hizo $branching$) en un mismo momento. Para elegir qué nodo es el próximo a ser analizado hay que establecer una \emph{estrategia de selección de nodo}. En el caso de este trabajo se optó por la política ``best-bound'' (elige el nodo cuya relajación tiene el valor del óptimo más conveniente). Por otra parte, para hacer $branching$ también suele establecerse una \emph{estrategia de selección de variable}. En este trabajo se eligió la política ``minimum-infeasibility'' (se elige la variable más cercana a un valor entero).\\

La idea intuitiva detrás de estas elecciones es que el recorrido en el árbol de $branching$ se encuentra con nodos de valor \emph{cercano} al óptimo y variables \emph{cercanas} a cumplir las condiciones de integralidad, por lo que el recorrido no tardará \emph{mucho} en encontrar el óptimo tras revisar un conjunto de nodos. Esta visión, de más está decirlo, es extremadamente ingenua y carece de sustento teórico conformante. Simplemente se eligieron esas políticas para que las tres técnicas algorítmicas sigan un recorrido análogo entre ellas, en vez de dejar que el $CPLEX$ tomara decisiones automáticamente. En la práctica este tipo de decisiones se toman luego de un extenso historial empírico que sugiera un favoritismo sobre determinadas estrategias; pero igualmente este enfoque puede ir variando diametralmente entre distintos MIPs.

\subsubsection{Más Comentarios: Implementación}

Se puede ver en el código presentado que se implementó un módulo \verb_Covers_ para el manejo de las desigualdades mochila para buscar cortes cover y un módulo \verb_Grafo_ para el manejo del grafo de conflictos para buscar cortes clique. Además, se implementó un módulo \verb_problemaCPLEX_ que actúa de intermediario entre aquéllos y el $CPLEX$. Esta modularización resultó de gran utilidad porque provee una independencia de implementación entre tres aspectos distintos pero que se relacionan entre sí. Un objeto \verb_problemaCPLEX_ no sólo guarda todo lo necesario para usar el $CPLEX$ (entorno, problema, status, etc), sino que además guarda un objeto \verb_Covers_ y un objeto \verb_Grafo_, de manera que hay una centralización hacia \verb_problemaCPLEX_ pero se garantiza independencia para sus propios métodos a \verb_Covers_ y \verb_Grafo_. Finalmente, en \verb_resolucion.cpp_ se tiene la interfaz entre \verb_problemaCPLEX_ y el usuario del $resolver$.\\

Se decide no ampliar la explicación del código porque son aspectos muy técnicos. El lector interesado puede tratar de comprender el código presentado usando como soporte la introducción a la \emph{CPLEX C Callable Library} expuesta en el  \emph{Apéndice}.

\newpage
\section{Resultados}

\newpage
\section{Conclusiones}

\subsection{Conclusiones del Trabajo}

\newpage
\subsection{Obsevaciones Finales}

\begin{itemize}

\item Con respecto al armado del grafo de conflictos, se encuentra que la segunda fase de armado casi nunca encuentra ejes nuevos. De todas las instancias que aparecen en la Tabla 3.1 (omitiendo cap6000, fast0507 y l152lav) sólo en tres se encuentran nuevos ejes: 1 en p0282, 128 en p0548 y 756 p2756 y, además, el algoritmo de partición sólo encuentra cliques de más de un nodo en lseu, p0201 y p0033 (además de las anteriores tres). Esto puede estar ocurriendo porque las matrices de restricciones de cada instancia o son muy ralas o justo hay desencuentro entre ejes del grafo y nodos que se analizan en la segunda fase (porque la segunda fase toma el subgrafo compuesto por nodos normales y nodos complemento, entonces podría ocurrir que en la primera fase se encuentre el eje ($x_i$,$x_j$) pero en la segunda fase el subgrafo contenga al eje ($\bar{x}_i$,$x_j$) con lo cual el eje encontrado no se analiza en la segunda fase). 


\item Con respecto a una decisión de implementación, se decidió a la hora de armar el grafo de conflictos usar un vector de listas de adyacencia en vez de manejar una matriz de adyacencia, por dos razones. En primer lugar, se intuye que la cantidad de ejes en el grafo de conflictos será muy pequeña, con lo cual tener una lista por cada nodo no introducirá mucho $overhead$. En segundo lugar, por la misma razón sí se tendrá una matriz de adyacencia muy rala con lo cual sí habrá mucho $overhead$ de memoria. Igualmente, podría analizarse qué pasa implementando el grafo de conflictos con la matriz a ver si esta modificación mejora los tiempos de la búsqueda de cortes clique (con la matriz saber si dos nodos son vecinos sea hace en $O(1)$ mientras que con la lista de adyacencias se tarda $O(|V(x_j)|)$ con $V(x_j)$ conjunto de vecinos del nodo $x_j$).


\item Se pueden introducir problemas numéricos con el $CPLEX$. Por ejemplo, en un momento durante este trabajo ocurría que $CPLEX$ agregaba cortes clique a pesar de que el grafo de conflictos no tenía ejes. Lo que ocurría era que $CPLEX$ admitía como solución factible casos donde $x_j = 1 + \epsilon$ ($\epsilon$ es una tolerancia) con lo cual según la implementación de ese entonces se agregaba el corte $x_j \leq 1$ (es decir, la implementación de ese entonces no consideraba el $\epsilon$). No sólo eso, sino que para obtener los valores complementarios $\bar{x}^*_j = 1.0 - x^*_j$ a veces ocurría que $\bar{x}^*_j$ terminaba siendo mayor a 1 ya que $x^*_j$ tomaba el valor de $-\epsilon$. Todas estas consideraciones no deben perderse de vista a la hora de la implementación.


\item Se encontró que $CPLEX$ puede crear distinos árboles de \emph{branching} para diferentes arquitecturas de procesador y hasta para distintas marcas de procesadores, a pesar de desactivar todos los parámetros del \verb_environment_ que controlan el recorrido de los árboles. Se recomienda realizar todos los tests siempre en una misma máquina.


\item La implementación aquí presentada puede perder vigencia ya que $CPLEX$ puede cambiar el nombre de las funciones. En caso de que esto ocurra, al compilar es probable que se obtenga un $warning$ que diga ``function is deprecated''. Esto quiere decir que la función se reemplaza por otra más nueva y quedará en desuso para futuras versiones del $CPLEX$ (es decir, los desarrolladores de la función ponen en el método desactualizado \verb@__attribute__((deprecated))@ para quitarle vigencia). En particular, en este trabajo ocurrió con la función \verb_CPXsetcutcallbackfunc_ cuya versión más actual es \verb_CPXsetusercutcallbackfunc_.

\end{itemize}


\newpage
\section{Referencias}


\end{document}
