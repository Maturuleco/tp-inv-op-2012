\subsubsection*{\underline{Tabla 3.1.0}}

\noindent Referencia de etiquetas para tablas de sección \emph{Resultados}.\\

{
\centering
\begin{tabular}{c p{10cm}}
\\
\hline
\textbf{Etiqueta} & \textbf{Descripción}\\
\hline
T & Tipo de Optimización (MIN minimización o MAX maximización)\\
\hline
ALG & Técnica algorítmica de resolución de MIP (BB, CB o BC)\\
& Y cortes usados: \\
& \hspace{5mm} - CC = cover (exacto y goloso) y clique\\
& \hspace{5mm} -  CQ = clique\\
& \hspace{5mm} -  CV = cover (exacto y goloso)\\
& \hspace{5mm} -  EC = cover exacto\\
& \hspace{5mm} -  EL = cover exacto y clique\\
& \hspace{5mm} -  GC = cover goloso\\
& \hspace{5mm} -  GL = cover goloso y clique\\
\hline
OPT & Indica si hubo timeout (N) o se encontró valor óptimo (Y) \\
& El caso Y contempla también el caso de frenado por tolerancia\\
\hline
MIP & Valor de mejor solución entera encontrada ($\mathbb{Z}$)\\
\hline
LP & Valor de solución óptima de la relajación en el nodo raíz ($\mathbb{R}$)\\
\hline
GR & $\%$ Gap del óptimo con respecto al nodo raíz\\
\hline
GF & $\%$ Gap alcanzado\\
\hline
TPO & Tiempo utilizado antes de la optimización (preparación de cortes), en segundos\\
\hline
TO & Tiempo utilizado durante toda la optimización, en segundos\\
\hline
TBC & Tiempo de optimización usado en buscar cortes, en segundos\\
\hline
VAR & Número de variables del MIP\\
\hline
RES & Número de restricciones del MIP\\
\hline
NOD & Número de nodos explorados con la técnica algorítmica\\
\hline
NOP & Número del nodo que tenía la mejor solución entera\\
\hline
NNE & Número de nodos que quedaron sin explorar\\
\hline
CVD & Número de cortes cover encontrados con algoritmo dinámico\\
\hline
CVG & Número de cortes cover encontrados con algoritmo goloso\\
\hline
CLI & Número de cortes clique encontrados\\
\hline
EGC & Número de ejes encontrados en el grafo de conflictos\\
\hline
\end{tabular}\\}
\vspace{1cm}

\noindent Por último, los valores reales fueron redondeados para mostrarlos únicamente con dos cifras decimales significativas.
