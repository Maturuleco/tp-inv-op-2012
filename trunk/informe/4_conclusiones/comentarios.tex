\subsection{Obsevaciones Finales}

\begin{itemize}

\item La performance del \emph{Branch \& Cut} no es la esperada, a pesar de que en estos algoritmos es donde mayor cantidad de cortes se encuentran para las instancias probadas. Se pudo ver que en ninguno de los casos el agregado de las heurísticas prometedoras hace más lento el algoritmo, aunque su verdadera contribución no parece ser tan alta. Este algoritmo podría modificarse con más restricciones (por ejemplo: no seguir buscando cortes para un mismo nodo del árbol $BC$ tras $X$ cantidad de búsquedas insatisfactorias) o podrían encontrarse nuevas heurísticas más exigentes para la decisión de buscar cortes o no. Asimismo, también puede estar ocurriendo que la cantidad de búsquedas insatisfactorias no sea significativa, sino que simplemente los cortes agregados recorten el espacio de búsqueda de una manera ínfima (no se encuentran facetas, sino restricciones que acotan mínimamente el poliedro de soluciones), con lo cual habría que considerar el agregado de nuevas familias de cortes al algoritmo $BC$. Todas estas variantes se plantean como consideraciones a futuro.


\item Con respecto al armado del grafo de conflictos, se encuentra que la segunda fase de armado casi nunca encuentra ejes nuevos. De todas las instancias que aparecen en la Tabla 3.1 (omitiendo cap6000, fast0507 y l152lav) sólo en tres se encuentran nuevos ejes: 1 en p0282, 128 en p0548 y 756 p2756 y, además, el algoritmo de partición sólo encuentra cliques de más de un nodo en lseu, p0201 y p0033 (además de las anteriores tres). Esto puede estar ocurriendo porque las matrices de restricciones de cada instancia o son muy ralas o justo hay desencuentro entre ejes del grafo y nodos que se analizan en la segunda fase (porque la segunda fase toma el subgrafo compuesto por nodos normales y nodos complemento, entonces podría ocurrir que en la primera fase se encuentre el eje ($x_i$,$x_j$) pero en la segunda fase el subgrafo contenga al eje ($\bar{x}_i$,$x_j$) con lo cual el eje encontrado no se analiza en la segunda fase). 


\item Con respecto a una decisión de implementación, se decidió a la hora de armar el grafo de conflictos usar un vector de listas de adyacencia en vez de manejar una matriz de adyacencia, por dos razones. En primer lugar, se intuye que la cantidad de ejes en el grafo de conflictos será muy pequeña, con lo cual tener una lista por cada nodo no introducirá mucho $overhead$. En segundo lugar, por la misma razón sí se tendrá una matriz de adyacencia muy rala con lo cual sí habrá mucho $overhead$ de memoria. Igualmente, podría analizarse qué pasa implementando el grafo de conflictos con la matriz a ver si esta modificación mejora los tiempos de la búsqueda de cortes clique (con la matriz saber si dos nodos son vecinos sea hace en $O(1)$ mientras que con la lista de adyacencias se tarda $O(|V(x_j)|)$ con $V(x_j)$ conjunto de vecinos del nodo $x_j$).


\item Se pueden introducir problemas numéricos con el $CPLEX$. Por ejemplo, en un momento durante este trabajo ocurría que $CPLEX$ agregaba cortes clique a pesar de que el grafo de conflictos no tenía ejes. Lo que ocurría era que $CPLEX$ admitía como solución factible casos donde $x_j = 1 + \epsilon$ ($\epsilon$ es una tolerancia) con lo cual según la implementación de ese entonces se agregaba el corte $x_j \leq 1$ (es decir, la implementación de ese entonces no consideraba el $\epsilon$). No sólo eso, sino que para obtener los valores complementarios $\bar{x}^*_j = 1.0 - x^*_j$ a veces ocurría que $\bar{x}^*_j$ terminaba siendo mayor a 1 ya que $x^*_j$ tomaba el valor de $-\epsilon$. Todas estas consideraciones no deben perderse de vista a la hora de la implementación.


\item Se encontró que $CPLEX$ puede crear distinos árboles de \emph{branching} para diferentes arquitecturas de procesador y hasta para distintas marcas de procesadores, a pesar de desactivar todos los parámetros del \verb_environment_ que controlan el recorrido de los árboles. Se recomienda realizar todos los tests siempre en una misma máquina.


\item La implementación aquí presentada puede perder vigencia ya que $CPLEX$ puede cambiar el nombre de las funciones. En caso de que esto ocurra, al compilar es probable que se obtenga un $warning$ que diga ``function is deprecated''. Esto quiere decir que la función se reemplaza por otra más nueva y quedará en desuso para futuras versiones del $CPLEX$ (es decir, los desarrolladores de la función ponen en el método desactualizado \verb@__attribute__((deprecated))@ para quitarle vigencia). En particular, en este trabajo ocurrió con la función \verb_CPXsetcutcallbackfunc_ cuya versión más actual es \verb_CPXsetusercutcallbackfunc_.

\end{itemize}
