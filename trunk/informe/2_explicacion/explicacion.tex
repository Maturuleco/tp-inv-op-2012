\section{Explicación}

\subsection{Cortes Cover}
\subsubsection{Cortes Cover: Explicación Teórica}

\subsubsection*{Definicion}
Una ``desigualdad mochila'' es una desigualdad con $a_j \geq 0$ ($1\leq j \leq n$) de la forma:

$$ \overset{n}{\underset{j=1}{\sum}} a_j\,x_j \leq b$$

En este ipo de desigualdades se pude buscar un ``cover''. Es decir, un conjunto $C \subseteq \{1,...,n\}$ de índices tales que:

$$ \overset{}{\underset{j \in C}{\sum}} a_j\,x_j > b$$

Si se encuentra el mismo, entonces se obtiene una nueva ``desigualdad cover'' válida:

$$ \overset{}{\underset{j \in C}{\sum}} x_j \leq |C| - 1$$

Donde $|C|$ es el cardinal del cover $C$ encontrado.

\subsubsection*{Corte Cover}

Un ``corte cover'', entonces, consiste en encontrar una desigualdad cover válida para toda solución entera del MIP pero que deje afuera soluciones fraccionarias. La idea es recibir una solución óptima de la relajación de un nodo $x^*$ como input y a partir de ella fabricar una desigualdad cover violada por $x^*$:\\

{
\centering
\begin{tabular}{c l}
\verb_INPUT_ & $x^*$\\
\verb_OUTPUT_ & Desigualdad cover violada por $x^*$\\
\end{tabular}\\
\vspace{5mm}
}

Para encontrar el corte cover en este trabajo se siguen los siguientes pasos:

\begin{enumerate}
\item Tomar una restricción original del MIP.
\item Traducirla a desigualdad mochila.
\item Búsqueda del cover.
\item Extensión del cover.
\item Reescribir el cover en función de variables originales.
\end{enumerate}

A continuación, la explicación detallada de cada paso.
\newpage

\begin{enumerate}
\item \underline{Tomar una restricción original del MIP:}\\
Cualquier restricción de la forma:
$$\overset{n}{\underset{j=1}{\sum}} a_j\;x_j \leq b$$

O, alternativamente, tomar una restricción de la siguiente forma:
$$\overset{n}{\underset{j=1}{\sum}} a_j\;x_j \geq b$$

Y multpilicar ambos lados por $-1$ para tener la forma de la primera inecuación (con ``$\leq$'').

\item \underline{Traducirla a desigualdad mochila:}\\
Con el conocimiento que las variables son binarias ($x_j\in\{0,1\}$) el primer paso consiste en traducir cada restricción original (con ``$\leq$'') del MIP en una desigualdad mochila. Sean los conjuntos:

$$N_+ = \{j: a_j > 0 \;\;\forall j\;\; 1\leq j \leq n\}$$
$$N_- = \{j: a_j < 0 \;\;\forall j\;\; 1\leq j \leq n\}$$

Y usando el cambio de variables:

$$x_j + \bar{x}_j = 1$$

Se puede hacer una traducción de la siguiente forma:

$$\overset{n}{\underset{j=1}{\sum}} a_j\;x_j \leq b$$
$$\Leftrightarrow$$
$$\overset{}{\underset{j \in N_+}{\sum}} a_j\;x_j - \overset{}{\underset{j \in N_-}{\sum}} |a_j|\;x_j \leq b$$
$$\Leftrightarrow$$
$$\overset{}{\underset{j \in N_+}{\sum}} a_j\;x_j - \overset{}{\underset{j \in N_-}{\sum}} |a_j|\;(1-\bar{x}_j) \leq b$$
$$\Leftrightarrow$$
$$\overset{}{\underset{j \in N_+}{\sum}} a_j\;x_j + \overset{}{\underset{j \in N_-}{\sum}} |a_j|\;\bar{x}_j \leq b + \overset{}{\underset{j \in N_-}{\sum}} |a_j|$$

De esta forma, en el caso de restricciones con variables binarias, cualquier inecuación puede traducirse a desigualdad mochila:

{
\centering
\begin{tabular}{p{4cm}p{4cm}p{4cm}}
$\overset{}{\underset{j \in N_+ \cup N_-}{\sum}} |a_j|\;\tilde{x}_j \leq \tilde{b}$
&
$\tilde{b} = b + \overset{}{\underset{j \in N_-}{\sum}} |a_j|$
&
$\tilde{x}_j = 
\begin{cases}
x_j & \text{ si } j \in N_+\\
\bar{x}_j & \text{ si } j \in N_-\\
\end{cases}$\\
\end{tabular}\\
\vspace{5mm}
}

\item \underline{Búsqueda del cover:}\\
Dada $x^*$ con $0\leq x_j \leq 1$ ($1\leq j \leq n$) solución óptima de la relajación, se busca un cover $C\subseteq \{1,...,n\}$ tal que:

{
\centering
\begin{tabular}{ccc}
$\overset{}{\underset{j \in C}{\sum}} x^*_j > |C|-1$ & y & $\overset{}{\underset{j \in C}{\sum}} \lfloor \tilde{a}_j \rfloor > \lceil \tilde{b} \rceil$\\
\end{tabular}\\
\vspace{5mm}
}

Este cover $C$ puede encontrarse resolviendo el MIP:

{
\centering
\begin{tabular}{rl}
minimizar & $\overset{n}{\underset{j=1}{\sum}} (1 - x^*_j) y_j$\\
sujeto a&\\
&$\overset{n}{\underset{j=1}{\sum}} \lfloor \tilde{a}_j \rfloor y_j \geq \lceil \tilde{b} \rceil + 1$\\
&$y_j \in \{0,1\}$\\
\end{tabular}\\
\vspace{5mm}
}

Si el funcional es mayor o igual a 1, no existe un cover violado por $x^*$. Si se encuentra $y^*$ tal que el valor del óptimo es menor a 1, los valores de $y^*$ hacen un cover violado. Sea $G=\{j: y^*_j = 1\}$ ($G$ es el cover $C$ buscado):

{
\centering
\begin{tabular}{cc}
$\overset{}{\underset{j \in G}{\sum}} \tilde{x}_j \leq |G|-1$ & $(< \overset{}{\underset{j \in G}{\sum}} x^*_j)$\\
\end{tabular}\\
\vspace{5mm}
}

Esta desigualdad es válida porque como muestra el MIP que la encuentra, es una violación a la restricción original del problema; y, además, la desigualdad válida es un plano de corte porque al agregarla se descarta la solución óptima $x^*$ ya que ésta viola el cover encontrado.

\item \underline{Extensión del cover:}

Una vez obtenido el cover $G$:

$$\overset{}{\underset{j \in G}{\sum}} \tilde{x}_j \leq |G|-1$$

Se puede extender de la siguiente forma:

$$E(G) = G \cup \{j: \tilde{a}_j \geq \tilde{a}_k \;\;\forall k\;\; \in G\}$$

Y queda el cover extendido:

$$\overset{}{\underset{j \in E(G)}{\sum}} \tilde{x}_j \leq |G|-1$$

\item \underline{Reescribir el cover en función de variables originales:}\\
Una vez que se consiguió un cover hay que recuperar las variables originales usando el reemplazo:

$$\tilde{x}_j = 
\begin{cases}
x_j & \text{ si } j \in N_+\\
1-x_j & \text{ si } j \in N_-\\
\end{cases}$$

Entonces:

$$\overset{}{\underset{j \in E(G)}{\sum}} \tilde{x}_j \leq |G| - 1$$
$$\Leftrightarrow$$
$$\overset{}{\underset{j \in E(G)\cap N_+}{\sum}}x_j + \overset{}{\underset{j \in E(G)\cap N_-}{\sum}} (1-x_j) \leq |G|-1$$

El cover quedará finalmente:

$$\overset{}{\underset{j\in E(G)}{\sum}} \tilde{x}_j \leq |G|-1$$
$$\Leftrightarrow$$
$$\overset{}{\underset{j\in E(G)\cap N_+}{\sum}} x_j - \overset{}{\underset{j \in E(G)\cap N_-}{\sum}} x_j \leq |G| - h -1$$

Con $h = |E(G)\cap N_-|$.
\end{enumerate}

\newpage
\subsubsection{Cortes Cover: Detalles de Implementación}

\newpage

\subsection{Cortes Clique}
\subsubsection{Cortes Clique: Explicación Teórica}

La idea central de este método es armar un grafo, llamado \textit{grafo de conflictos}, que respresenta relaciones l\'ogicas entre las variables del problema lineal. Dicho grafo tiene un v\'ertice por cada variable ($x_i$) y su complemento ($\bar{x}_i = 1 - x_i$), y un eje entre dos v\'ertices cuando a lo sumo una de las dos variables que comparten el eje puede estar en 1 en una soluci\'on \'optima. Hay cuantro relaciones l\'ogicas entre las variables (recordemos que las variables son binarias):
\begin{eqnarray*}
x_i = 1 \Rightarrow x_j = 0 & \Longleftrightarrow & x_i + x_j \leq 1 \\
x_i = 0 \Rightarrow x_j = 0 & \Longleftrightarrow & (1 - x_i) + x_j \leq 1 \\
x_i = 1 \Rightarrow x_j = 1 & \Longleftrightarrow & x_i + (1 - x_j) \leq 1 \\
x_i = 0 \Rightarrow x_j = 1 & \Longleftrightarrow & (1 - x_i) + (1 - x_j) \leq 1 
\end{eqnarray*}

Seguro una variable y su complemento est\'an unidos por un eje ya que exactamente una de las dos tiene que ser igual a 1 (en realidad es una relaci\'on m\'as fuerte). Entonces, se construye un grafo $G = (V,E)$ utilizando t\'ecnicas para deducir relaciones entre variables. Estas t\'ecnicas son llamadas de \textit{probing}: b\'asicamente es setear alguna variable binaria en una de sus cotas y estudiar las consecuencias. Si se asigna una variable en una de sus cotas y el problema se vuelve infactible, esto quiere decir que la variable necesariamente tiene que estar en la cota opuesta.

Se van a desarrollar brevemente dos t\'ecnicas utilizadas en el trabajo para deducir relaciones entre variables para agregar ejes al \textit{grafo de conflictos}. Una primera fase relativamente simple para agregar ejes y una segunda fase m\'as fuerte para intentar completar con m\'as ejes.

\subsubsection*{Primera fase}

Sea $S = \{x\in \{0,1\}^n : Ax \leq b\}$ donde $Ax \leq b$ representa las restricciones lineales del problema a resolver. Se define el siguiente conjunto:
$$S_{x_i=v_i,x_j=v_j} = \{x\in S : x_i = v_i, x_j = v_j\}$$ 
donde $v_i,v_j\in \{0,1\}$ y $x_i \neq x_j$. Es decir, las soluciones del problema tales que $x_i$ y $x_j$ est\'an seteados en $v_i$ y $v_j$ respectivamente. Como calcular este conjunto es tan dif\'icil como resolver el problema original, se trabaja con la relajaci\'on $S'_{x_i=v_i,x_j=v_j}$ de $S_{x_i=v_i,x_j=v_j}$ (pues $S'_{x_i=v_i,x_j=v_j} \supseteq S_{x_i=v_i,x_j=v_j}$). \\

Luego, sea $L^r_{x_i=v_i,x_j=v_j} = min\{a_rx : x\in S'_{x_i=v_i,x_j=v_j}\}$ donde $a_r$ es la \textit{r-\'esima} fila de $A$. De $L^r$ se puede deducir lo siguiente:
\begin{itemize}
\item $L^r_{x_i=1,x_j=1} > b_r \Longrightarrow x_i + x_j \leq 1$ es v\'alido para $S$ y $(i,j)\in E$
\item $L^r_{x_i=1,x_j=0} > b_r \Longrightarrow x_i + (1-x_j) \leq 1$ es v\'alido para $S$ y $(i,\bar{j})\in E$ 
\item $L^r_{x_i=0,x_j=1} > b_r \Longrightarrow (1-x_i) + x_j \leq 1$ es v\'alido para $S$ y $(\bar{i},j)\in E$ 
\item $L^r_{x_i=0,x_j=0} > b_r \Longrightarrow (1-x_i) + (1-x_j) \leq 1$ es v\'alido para $S$ y $(\bar{i},\bar{j})\in E$
\end{itemize}

Utilizar la relajaci\'on $S'_{x_i=v_i,x_j=v_j}$ puede ser costosa, as\'i que se puede usar otra relajaci\'on m\'as d\'ebil pero con un esfuerzo computacionalmente menor. La relajaci\'on utilizada es la siguiente: usar simplemente las cotas de las variables. Para cada fila $r$ se define: \\ 
$$B^+_r = \{j: a_{rj} > 0 \text{ con } 1\leq j \leq n\}$$
$$B^-_r = \{j: a_{rj} < 0 \text{ con } 1\leq j \leq n\}$$

Entonces, se toma la siguiente relajaci\'on: 

$$L^r_{x_i=v_i,x_j=v_j} = \overset{}{\underset{k\in B^-_r,k\neq i,k\neq j}{\sum}} a_{rk} + a_{ri}v_i + a_{rj}v_j$$

As\'i, para cada restricci\'on $r$ se hacen todas las combinaciones con $x_i = v_i$ y $x_j = v_j$, $v_i,v_j\in \{0,1\}$ y se calcula el $L^r$ correspondiente para saber si $L^r > b_r$. En tal caso, se agrega el eje correspondiente al grafo de conflictos.

\subsubsection*{Segunda fase}

Utilizando las cotas de las variables y el grafo ya armado en la primera fase se puede conseguir una relajaci\'on m\'as fuerte. Se considera un subconjunto de v\'ertices $U \subseteq V$ que denotan las variables en $B^-_r$ y lo complementos de variables en $B^+_r$ de la fila $r$ de $A$. Definimos $w_k = |a_{rk}|$ para cada $k\in U$. Entonces, una cota inferior de $L^r$ para el m\'inimo valor del lado izquierdo de la fila $r$ se obtiene resolviendo el problema del conjunto independiente m\'as pesado del subgrafo inducido por los vértices de $U$:
\begin{center}
$\zeta^r = \sum_{k\in B^+_r}a_{rk} - max\{\sum_{k\in U}w_k\;z_k : z\quad$conjunto independiente de$\quad G(U)\}$
\end{center}

donde $G(U)$ es el subgrafo de $G$ inducido por $U$ y $z\in \{0,1\}^{|U|}$ el vector que caracteriza el conjunto independiente de $G(U)$ (es decir, $z_k = 1$ si $k$ pertenece al conjunto independiente y $z_k = 0$ si no). Se agrega la sumatoria de los $a_{rk}$ con $k\in B^+_r$ pues se usan los complementos de las variables con \'indices en $B^+_r$. 

Como resolver este nuevo problema es $NP$-hard, se resuelve una relajaci\'on del conjunto independiente m\'as pesado observando que si un grafo consiste en subgrafos completos disjuntos el conjunto independiente \'optimo se consigue eligiendo el v\'ertice m\'as pesado de cada subgrafo completo. Entonces, la idea es particionar los v\'ertices de $G(U)$ en un conjunto de cliques, utilizando alg\'un algoritmo que no sea muy costoso, y elegir el v\'ertice m\'as pesado de cada clique.

Una vez que la partici\'on en cliques está hecha, se busca un cota inferior para $L^r_{x_i=v_i,x_j=v_j}$ eligiendo el v\'ertice m\'as pesado de cada clique sujeto a que $x_i = v_i$ y $x_j = v_j$. Esto se hace para cada combinaci\'on de $v_i,v_j\in \{0,1\}$ teniendo cuidado de ignorar los casos en que haya un eje en el grafo de conflictos con dicha combinaci\'on.\\

{\footnotesize\underline{Observación:} para más detalles ver el \emph{paper} citado en $[1]$.}

\subsubsection*{Corte Clique}

Un ``corte clique'', consiste en encontrar una o m\'as cliques maximales en el grafo de conflictos armado que representen una desigualdad válida para toda solución entera del MIP pero que deje afuera soluciones fraccionarias. La idea es recibir la solución óptima de la relajación de un nodo $x^*$ como input y a partir de ella deducir una desigualdad clique violada por $x^*$:\\

{
\centering
\begin{tabular}{c l}
\verb_INPUT_ & $x^*$\\
\verb_OUTPUT_ & Desigualdad clique violada por $x^*$\\
\end{tabular}\\
\vspace{5mm}
}

Entonces la idea es buscar una (ó más) clique maximal $K$ en el grafo de conflictos tal que: 

$$\sum_{j\in K}\tilde{x}^*_j > 1$$

donde 
$\tilde{x}^*_j = 
\begin{cases}
\bar{x}^*_j & \text{ si j es complemento en el grafo}\\
x^*_j & \text{ si no}\\
\end{cases}$\\

Si se encuentra $K$, se puede agregar el siguiente corte clique v\'alido al conjunto de restricciones del MIP:

$$\sum_{j\in K}\tilde{x}_j \leq 1$$

Para encontrar un corte clique en este trabajo se siguen los siguientes pasos:

\begin{enumerate}
\item Armar el grafo de conflictos con los métodos antes explicados.
\item Buscar una o m\'as cliques maximales en el grafo, violadas por la relajaci\'on lineal.
\item Reescribir la desigualdad clique en función de las variables originales.
\end{enumerate}

A continuaci\'on se explica brevemente cada paso.
\newpage

\begin{enumerate}[ 1{)} ]
\item \underline{Armar el grafo de conflictos con las fases antes explicadas:}\\

Por cada restricci\'on del MIP se busca agregar ejes al grafo de conflictos probando cada combinaci\'on de $v_i$ y $v_j$ asignadas a cada variable con coeficiente distinto de 0 de la restricci\'on tanto en la primera fase como en la segunda. Luego de las dos fases queda armado el grafo de conflictos con las relaciones l\'ogicas v\'alidas entre las variables.

\item \underline{Buscar una o m\'as cliques maximales en el grafo, violadas por la relajaci\'on lineal:}\\

Con el óptimo de la relajaci\'on lineal obtenido y el grafo de conflictos armado, se buscan una o m\'as cliques maximales (con algún algoritmo heurísitico) tal que los valores de la relajaci\'on violen la desigualdad clique ($\sum_{j\in K}\tilde{x^*_j} > 1$ donde $K$ es la clique).

\item \underline{Reescribir la desigualdad clique en función de las variables originales:}\\

Una vez conseguido el corte clique (llamemos $K$ a la clique) hay que recuperar las variables originales usando el reemplazo:
$\tilde{x}_j = 
\begin{cases}
1-x_j & \text{ si } \bar{j}\in K\\
x_j & \text{ si } j\in K\\
\end{cases}$\\

Entonces:

$$\sum_{j\in K}\tilde{x}_j \leq 1$$
$$\Longleftrightarrow$$
$$\sum_{j\in K}x_j + \sum_{\bar{j}\in K}(1-x_j) \leq 1$$

El corte clique finalmente queda:

$$\sum_{j\in K}x_j - \sum_{\bar{j}\in K}x_j \leq 1 - k$$

donde $k = |\{\bar{j} : \bar{j}\in K\}|$

\end{enumerate}

\newpage
\subsubsection{Cortes Clique: Detalles de Implementación}

\newpage

\subsection{Más Comentarios}

Además de lo comentado anteriormente, hay que aclarar que a la hora de recorrer el árbol de subproblemas derivados del MIP original hay dos puntos clave en el que hay que elegir una estrategia de recorrido. El $CPLEX$ puede tener varios nodos (subproblemas derivados del MIP) activos (aún no se calculó su relajación lineal y no se hizo $branching$) en un mismo momento. Para elegir qué nodo es el próximo a ser analizado hay que establecer una \emph{estrategia de selección de nodo}. En el caso de este trabajo se optó por la política ``best-bound'' (elige el nodo cuya relajación tiene el valor del óptimo más conveniente). Por otra parte, para a hacer $branching$ también suele establecerse una \emph{estrategia de selección de variable}. En este trabajo se eligió la política ``minimum-infeasibility'' (se elige la variable más cercana a un valor entero).\\

La idea intuitiva detrás de estas elecciones es que el recorrido en el árbol de $branching$ se encuentra con nodos de valor \emph{cercano} al óptimo y variables \emph{cercanas} a cumplir las condiciones de integralidad, por lo que el recorrido no tardará \emph{mucho} en encontrar el óptimo tras revisar un conjunto de nodos. Esta visión, de más está decirlo, es extremadamente ingenua y carece de sustento teórico conformante. Simplemente se eligieron esas políticas para que las tres técnicas algorítmicas sigan un recorrido análogo entre ellas, en vez de dejar que el $CPLEX$ tomara decisiones automáticamente. En la práctica este tipo de decisiones se toman luego de un extenso historial empírico que sugiera un favoritismo sobre determinadas estrategias; pero igualmente este enfoque puede ir variando diametralmente entre distintos MIPs.
