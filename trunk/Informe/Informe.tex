%%	SECCION documentclass																									 %%	
%%---------------------------------------------------------------------------%%
\documentclass[a4paper]{report}
%%---------------------------------------------------------------------------%%
%%	SECCION usepackage																											 %%	
%%---------------------------------------------------------------------------%%
\usepackage{amsmath, amsthm}
\usepackage{amsfonts}%
\usepackage{amssymb}%
\usepackage{caratula}
\usepackage{hyperref} %para que haya hiperv�nculos
\usepackage{graphicx} % Para el logo magico!
\usepackage{alltt} % que es este paquete???

\usepackage[spanish]{babel} 
\usepackage[T1]{fontenc}
\usepackage{textcomp}
\usepackage[utf8]{inputenc}


\oddsidemargin 0cm
\headsep -1.4cm
\textwidth=16.5cm
\textheight=23cm
%\makeindex



%%---------------------------------------------------------------------------%%
%%	SECCION document	                                                       %%	
%%---------------------------------------------------------------------------%%
\begin{document}
	\setcounter{page}{0} %para que numere a la caratula como p�gina 0 y a las demas a partir de 1

%%---- Caratula -------------------------------------------------------------%%
\title{Investigación Operativa}
\author{Matías Pérez\thanks{This is for making an acknowledgement.}
\\Universidad de Buenos Aires, Argentina}
\date{14 April, 2009}

\materia{Investigación Operativa}
\titulo{Trabajo Práctico}
\subtitulo{Resolución de problemas lineales.}
\fecha{diciembre, 2012}


\integrante{Fabrizio Borghini}{???/??}{fabriborghini@gmail.com}
\integrante{Sebastián ???????}{???/??}{cobyto@gmail.com}
\integrante{Matías Pérez}{002/05}{elmaildematiaz@gmail.com}
%\grupo{}
%\nombreGrupo{}
%\resumen{}

\maketitle

%\tableofcontents

\newpage

\clearpage

\include{solucion}

\end{document}
