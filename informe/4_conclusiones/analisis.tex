\subsection{Conclusiones del Trabajo}

\begin{itemize}

\item La performance del \emph{Branch \& Cut} no es la esperada, a pesar de que en estos algoritmos es donde mayor cantidad de cortes se encuentran para las instancias probadas. Se pudo ver que en ninguno de los casos el agregado de las heurísticas prometedoras hace más lento el algoritmo, aunque su verdadera contribución no parece ser tan alta. Este algoritmo podría modificarse con más restricciones (por ejemplo: no seguir buscando cortes para un mismo nodo del árbol $BC$ tras $X$ cantidad de búsquedas insatisfactorias) o podrían encontrarse nuevas heurísticas más exigentes para la decisión de buscar cortes o no. Asimismo, también puede estar ocurriendo que la cantidad de búsquedas insatisfactorias no sea significativa, sino que simplemente los cortes agregados recorten el espacio de búsqueda de una manera ínfima (no se encuentran facetas, sino restricciones que acotan mínimamente el poliedro de soluciones), con lo cual habría que considerar el agregado de nuevas familias de cortes al algoritmo $BC$. Todas estas variantes se plantean como consideraciones a futuro.

\item Como se observa en los resultados obtenidos, este tipo de problemas de optimización depende mucho de cada instancia ya que en nuestro caso los cortes agregados no parecían muy significativos a la hora de comparar resultados y tiempos. Justamente, en las instancias que se corrieron en la mayoría de los casos, los resultados de los algoritmos de $BC$ y $CB$ (que en teoría fueron ideados para mejorar los algoritmos ya existentes) no diferían mucho con los del $BB$. Esto resultó ser una sorpresa porque se creía que, por lo visto en la materia, los cortes implementados iban a ser más eficientes. Esto da la pauta que se requiere mucho trabajo de experimentación e incluso de optimización en la búsqueda de cortes. Además, aquí sólo se implementan dos familias de cortes genéricos cuando quizás los más práctico es analizar varias familias de cortes y estudiar cortes específicos para cada problema.

\item Otra consideración, que no mejora el panorama, es que para encontrar cortes en muchos casos complica aún más la complejidad original del MIP propiamente dicho. En este caso para buscar cortes cover hay que resolver un problema $NP$ (o \emph{pseudo-polinomial}) que es el problema de la mochila sin repeticiones; en el caso de los cortes clique, hay que encontrar cliques maximales y habría que resolver el problema de Conjunto Independiente Pesado Máximo para el armado del grafo. Sin embargo, para paliar estas dificultades es que se hacen algoritmos heurísticos aproximados que presentan una dualidad: suelen ser algoritmos rápidos, pero no siempre se encuentran las mejores soluciones.

\item No se pudo abstraer una visión general tras los resultados obtenidos, porque cada instancia tenía sus peculiaridades. Este fenómeno es muy común en la búsqueda de soluciones en el área de \emph{Optimización Combinatoria} y es por eso que hay tanta experimentación para este tipo de problemas. En realidad lo que ocurre es que hay una carencia teórica que no permite elaborar algoritmos eficientes y correctos.

\end{itemize}
